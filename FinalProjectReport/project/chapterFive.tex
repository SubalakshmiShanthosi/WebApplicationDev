\chapter{Forms in Django}
\section{Introduction}
\paragraph{} Django provides a rich framework to facilitate the creation of forms and the manipulation of form data.
\begin{itemize}
	\item \textbf{The basics:} 
	\begin{itemize}
		\item Overview 
		\item Form API 
		\item Built-in fields 
		\item Built-in widgets
	\end{itemize}
   \item \textbf{Advanced:}
   \begin{itemize}
   	\item Forms for models 
   	\item Integrating media  
   	\item Formsets 
   	\item Customizing validation
   \end{itemize}
\end{itemize}
\subsection{Django's role in forms}
Django handles three distinct parts of the work involved in forms:\\
\begin{itemize}
	\item Preparing and Restructuring data to make it ready for rendering.
	\item Creating HTML forms for the data.
	\item Receiving and Processing submitted forms and data from the client.
\end{itemize}
\subsection{Forms in Django}
\subsubsection{ The Django \textbf{Form} class:}
Django Model describes the logical \textbf{structure} of an object, it's \textbf{behavior}, and the way its parts are represented to us, a Form class describes a form and determines how it works and appears.\\

In a similar way that a model class's fields map to database fields, a form class's fields map to HTML form <input> elements. (A \textbf{ModelForm} maps a model class's fields to HTML form <input> elements via a Form; this is what the Django admin is based upon.)\\

A form's fields are themselves classes; they manage form data and perform validation when a form is submitted. A DateField and a FileField handle very different kinds of data and have to do different things with it.\\

A form field is represented to a user in the browser as an HTML "\textbf{widget}" - a piece of user interface machinery. Each field type has an appropriate default Widget class, but these can be overridden as required.

\subsubsection{Instantiating, processing, and rendering forms}
While \textbf{rendering} a form:
\begin{itemize}
	\item Get hold of it in the view (fetch it from the database, for example).
	\item Pass it to the template context.
	\item Expand it to HTML markup using template variables.	
\end{itemize}
When we \textbf{instantiate} a form, we can opt to leave it empty or pre-populate it, for example with:
\begin{itemize}
	\item Data from a saved model instance (as in the case of admin forms for editing).
	\item Data that we have collated from other sources.
	\item Data received from a previous HTML form submission.	
\end{itemize}
\newpage
\section{Form Field Types}
\paragraph{}Each model field has a corresponding default form field. For example, a CharField on a model is represented as a CharField on a form. A model ManyToManyField is represented as a MultipleChoiceField.

\begin{table}[]
	\begin{tabular}{|l|l|ll}
		\cline{1-2}
		\textbf{Model Field} & \textbf{Form Field}       &  &  \\ \cline{1-2}
		AutoField            & Not represented in form   &  &  \\ \cline{1-2}
		BigAutoField         & Not represented in form   &  &  \\ \cline{1-2}
		BigIntegerField      & IntegerField              &  &  \\ \cline{1-2}
		BinaryField          & CharField - True or false &  &  \\ \cline{1-2}
		BooleanField         & BooleanField              &  &  \\ \cline{1-2}
		CharField            & CharField                 &  &  \\ \cline{1-2}
		DateField            & DateField                 &  &  \\ \cline{1-2}
		ForeignKey           & ModelChoiceField          &  &  \\ \cline{1-2}
		ManyToManyField      & ModelMultipleChoiceField  &  &  \\ \cline{1-2}
		TextField            & CharField                 &  &  \\ \cline{1-2}
	\end{tabular}
\caption{Conversion of Model to Form Fields }
\label{tab:my-table}
\end{table}
\subsection{Accessing the fields from the form}
Using command \textbf{Form.fields}
\begin{lstlisting}[language=python,numbers=none]
>>> f.as_table().split('\n')[0]
'<tr><th>Name:</th><td><input name="name" type="text" value="instance" required></td></tr>'
>>> f.fields['name'].label = "Username"
>>> f.as_table().split('\n')[0]
'<tr><th>Username:</th><td><input name="name" type="text" value="instance" required></td></tr>
\end{lstlisting}

\subsection{Outputting forms as HTML}
The second task of a Form object is to render itself as HTML. To do so, simply print it:
\begin{itemize}
	\item \textbf{Form.as\_p()}:\\
	as\_p() renders the form as a series of \textbf{<p>} tags, with each <p> containing one field.
	\item \textbf{Form.as\_ul()}: Similarly,Render with series of \textbf{<ul>} tags.
	\item \textbf{Form.as\_table()}:Renders as HTML table.
\end{itemize}
\newpage
\section{Forms in FormMason}
Form with inplace vaildation based on the field type and additional parameters such as max-length,min-length,required etc. specified on Form Fieldtype constructor.\\
\begin{lstlisting}[language=python,numbers=none]
# Form class which inherits django forms. Consists of four fields Name,Age,Address and Gender
from django import forms
class SampleForm(forms.Form):
	name = forms.CharField()
	age = forms.IntegerField()
	address = forms.CharField(required=False)
	gender = forms.ChoiceField(choices=(('M', 'Male'),('F','Female')))
\end{lstlisting}
\subsubsection{Use of Ordered Dict Type in Django}
Unlike the traditional Dictonary - the fields of Form are of type Ordered Dictionary.\\
This is to \textbf{preserve} the order of insertion of field while writing forms.py.

\subsection{Customised Form}
New Dynamic Form(NewDynamicFormForm class) with the following fields as class members:
\begin{enumerate}
\item \textbf{form\_pk} : CharField - required : False, HiddenInput.
\item \textbf{title} : CharField - required : True, Normal Input.
\item \textbf{schema} : CharField - required : True, Normal Input.
\item Method \textbf{clean\_schema} : To trying loading json schema and returning schema.
\end{enumerate}
\newpage
\begin{lstlisting}[language=python,numbers=none]
 import json
 from django import forms
 from django.db.models.functions import Lower,Upper
 
 # Form fields - JSON data and title.
 class NewDynamicFormForm(forms.Form):
 	form_pk=forms.CharField(widget=forms.HiddenInput(),required=False)
 	title = forms.CharField()
 	schema = forms.CharField(widget=forms.Textarea()) 
 # Method to load the JSON schema
 def clean_schema(self):
 	schema = self.cleaned_data["schema"]
 	schema = json.loads(schema)
 	return schema
\end{lstlisting}
$\bullet${Alteration in views and template}
\begin{lstlisting}[language=python,numbers=none]
#Create Edit Form View to load data from JSON into created model
class CreateEditFormView(FormView):
	form_class = NewDynamicFormForm
	template_name = "create_edit_form.html"
	def get_initial(self):
		if "form_pk" in self.kwargs:
			form = FormSchema.objects.get(pk=self.kwargs["form_pk"])
			initial = {
				"form_pk": form.pk,
				"title": form.title,
				"schema": json.dumps(form.schema)
			}
		else:
			initial = {}  # Ensure backward compactibility of old form layout
	return initial
# Get context data - Form context
def get_context_data(self, **kwargs):
	ctx = super(CreateEditFormView, self).get_context_data(**kwargs)
	if "form_pk" in self.kwargs:
		ctx["form_pk"] = self.kwargs["form_pk"]
	return ctx
# Save old or new form fields.
def form_valid(self, form):
	cleaned_data = form.cleaned_data
	if cleaned_data.get("form_pk"):
		old_form = FormSchema.objects.get(pk=cleaned_data["form_pk"])
		old_form.title = cleaned_data["title"]
		old_form.schema = cleaned_data["schema"]
		old_form.save()
	else:
		new_form = FormSchema(title=cleaned_data["title"],schema=cleaned_data["schema"])
		new_form.save()
		return HttpResponseRedirect(reverse("home"))
\end{lstlisting}
\begin{lstlisting}[language=html,numbers=none]
<!-- Changes to include link for edit form -- >
<! -- create_edit_form.html-->


<h1>Create/Edit Form</h1>

<li>
<form action="" method="post">

</li>

<li>
<form action="" method="post">

</li>
{{ form.as_p }}
<li>
<input type="submit" value="Create Form" />
</li>
</form>

\end{lstlisting}